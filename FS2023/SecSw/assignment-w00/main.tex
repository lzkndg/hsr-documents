\documentclass[11pt, a4paper]{article}
\usepackage[utf8]{inputenc}
\usepackage{graphicx}
\graphicspath{ {./res/} }
\usepackage{hyperref}
\title{Secure Software, W00 Python venv}
\author{Luzia Kündig}
\begin{document}

\maketitle

\subsubsection*{How do you exit your current virtual environment?}

By typing \verb|deactivate| inside the shell.

\subsubsection*{What would you do, if the software is a python2 software instead of python3?}

Either run it using a Python2 interpreter or convert it to Python3. The different major versions are not compatible. There are methods to call python2 scripts from within python3 code, but they rely on starting a new process with the right interpreter.

\subsubsection*{Explain the purpose of Pipfile and Pipfile.lock from the GitHub Repo}

The \verb|Pipfile| contains concrete requirements for a python application and is populated manually by the developer. These requirements include a package's source as well as loose version constraints. It also allows grouping of referenced packages, to separate dev-dependencies for example. 

Detailed information of the environment, meaning all installed packages with pinned versions and more details, will be stored in a \verb|pipfile.lock| which is generated using the command \verb|pipenv lock|.

Source: \url{https://github.com/pypa/pipfile}

\end{document}