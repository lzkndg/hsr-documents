\chapter{Haftpflichtrecht, Produkthaftpflicht}
\begin{description}
  \item[Verschuldenshaftung (OR 41)] Voraussetzungen: Schaden (unfreiwillige Vermögenseinbusse), Adäquater Kausalzusammenhang (Zusammenhang von schuldhaftem Verhalten und Schaden), Widerrechtlichkeit (Verletzung der Persönlichkeit oder Eigentum ohne Rechtfertigungsgrund), Verschulden (Fahrlässigkeit oder Vorsatz)
  
  \item[Kausalhaftung] gilt in Ausnahmen, ohne dass ein direktes Verschulden nötig ist. Geschäftsherrenhaftung (OR 55), Tierhalterhaftung (OR 56), Werkeigentümerhaftung, Haftung des Grundeigentümers, Haftung des Familienoberhaupts, haftung Urteilsunfähiger, Staats- und Beamtenhaftung, Produkthaftung.
  
  \textbf{Milde} Kausalhaftung kann mittels Sorgfaltsbeweis entlastet werden. 
  \textbf{Scharfe} Kausalhaftung ist zwingend. Beispiel: Verunreinigung eines Gewässers durch den Grundeigentümer, Werkeigentümer trotz ordnungsgemässem Unterhalt. Befreiungsbeweis bei fehlender Kausalität ist möglich.
  
  \item[Gefährdungshaftung] Qualifizierte Gefährdung durch Vorrichtung oder Tätigkeit. Im Unterschied zur Haftung aus unerlaubter Handlung kommt es nicht auf die Widerrechtlichkeit der Handlung oder ein Verschulden des Schädigers an, sondern wird bspw. alleine schon durch den Betrieb (z.B. Motorfahrzeug) begründet. Eine potenzielle Gefährdung anderer ist unvermeidbar. Beispiele: Betrieb eines Motorfahrzeugs, Jagdbetrieb, Flugbetrieb, Betrieb einer Atomanlage (in Spezialgesetzen geregelt).
  
  \item[Vertragliche Haftung]
\end{description}

\section{Produktehaftpflicht}
Produktehaftung ist das Einstehenmüssen des Herstellers eines Produktes für Schäden, die aus dem Gebrauch eines in Verkehr gebrachten fehlerhaften Produktes entstehen.

Gehaftet wird nur für Mangelfolgeschäden, somit nicht für Schäden am Produkt selbst. Produktschäden fallen unter die kauf- oder werkvertraglichen Sachgewährleistung (Garantie).
\vspace{3mm}

\textbf{PrHG 1 - Grundsatz}

\textit{1 Die herstellende Person haftet für den Schaden, wenn ein fehlerhaftes Produkt dazu führt, dass eine Person getötet oder verletzt wird (Personenschaden), eine Sache beschädigt oder zerstört wird, die nach ihrer Art gewöhnlich zum privaten Gebrauch bestimmt und vom Geschädigten hauptsächlich privat verwendet worden ist (Sachschaden).}

\textit{2 Die Herstellerin haftet \textbf{nicht} für den Schaden am fehlerhaften Produkt.} Führt bspw. ein fehlerhaftes Kletterseil zu einem Sportunfall, geht es vorliegend nicht um den Ersatz des fehlerhaften Produktes.
\vspace{3mm}

\textbf{PrHG 2 - Begriff der Herstellerin}

\textit{1 Als Herstellerin im Sinne dieses Gesetzes gilt:
a. die Person, die das Endprodukt, einen Grundstoff oder ein Teilprodukt hergestellt hat;
b. jede Person, die sich als Herstellerin ausgibt, indem sie ihren Namen, ihr Warenzeichen oder ein anderes Erkennungszeichen auf dem Produkt anbringt;
c. jede Person, die ein Produkt zum Zweck des Verkaufs, der Vermietung, des Mietkaufs oder einer andern Form des Vertriebs im Rahmen ihrer geschäftlichen Tätigkeit einführt; dabei bleiben abweichende Bestimmungen in völkerrechtlichen Verträgen vorbehalten.}

\textit{2 Kann die Herstellerin des Produkts nicht festgestellt werden, so gilt jede Person als Herstellerin, welche das Produkt geliefert hat, sofern sie dem Geschädigten nach einer entsprechenden Aufforderung nicht innerhalb einer angemessenen Frist die Herstellerin oder die Person nennt, die ihr das Produkt geliefert hat.}
\vspace{3mm}

\noindent
\textbf{PrHG 4 - Begriff des Fehlers}

\textit{1 Ein Produkt ist fehlerhaft, wenn es nicht die Sicherheit bietet, die man unter Berücksichtigung aller Umstände zu erwarten berechtigt ist; insbesondere sind zu berücksichtigen:
a. die Art und Weise, in der es dem Publikum präsentiert wird;
b. der Gebrauch, mit dem vernünftigerweise gerechnet werden kann;
c. der Zeitpunkt, in dem es in Verkehr gebracht wurde.}

\textit{2 Ein Produkt ist nicht allein deshalb fehlerhaft, weil später ein verbessertes Produkt in Verkehr gebracht wurde.}

\noindent
\subsubsection*{Voraussetzungen der Haftung}
Sämtliche Punkte müssen erfüllt sein, damit die Haftung des Herstellers nach PrHG gegeben ist.

\begin{enumerate}
  \item Er ist \textbf{Hersteller / Importeur / Händler} nach \textit{PrHG 2}
  \item Es handelt sich um ein \textbf{Produkt} nach \textit{PrHG 3}
  \item Das Produkt ist im Sinne von \textit{PrHG 4} \textbf{fehlerhaft.}
  \item Es verursachte einen Schaden im Sinne von \textit{PrHG 1} und der \textbf{Schaden ist grösser als CHF 900}.
  \item Zwischen dem fehlerhaften Produkt und dem Schaden besteht ein \textbf{adäquater Kausalzusammenhang} (Ursache - Wirkung).
\end{enumerate}

Merke: die Haftung kann nicht durch Vereinbarungen wegbedungen werden. Verjährungsfrist 3 Jahre, Verwirkungsfrist 10 Jahre.
\vspace{3mm}

Folgende Gründe erlauben dem Hersteller, eine bestehende Haftung trotzdem abzuwehren.

\begin{itemize}
  \item Produkt nicht in Verkehr gebracht
  \item Fehler bei Inverkehrssetzung noch nicht vorhanden
  \item Produkt für den Verkauf oder eine Art des Vertriebs mit wirtschaftlichem Zweck oder im Rahmen der beruflichen Tätigkeit hergestellt bzw. Vertrieben
  \item Fehler ist darauf zurück zu führen, dass das Produkt verbindlichen, hoheitlich erlassenen Vorschriften entspricht
  \item Fehler konnte nach dem Stand der Wissenschaft und Technik im Zeitpunkt, in dem das Produkt in Verkehr gebracht wurde, nicht erkannt werden (Ausnahmen in PrHG 5 Abs. 1bis)
  \item Hersteller eines Grundstoffes oder Teilprodukts, wenn der Fehler erst mit der weiteren Verarbeitung entstanden ist
\end{itemize}

\subsubsection*{Bundesgerichtsentscheid (133 III 81) zum Fehlerbegriff}
Vorliegend geht um eine gläserne Kaffeekanne, die plötzlich explodiert ist und der Geschädigten / Klägerin / Konsumentin damit schwere Handverletzungen zugefügt hat. Die Schadenersatzklage richtete sich gegen den Importeur der Kanne. Die Geschädigte unterlag jeweils bei den Vorinstanzen, weil sie bei der Kanne weder einen Konstruktions-, noch einen Fabrikations- oder Instruktionsfehler nachweisen konnte.

Gemäss Bundesgericht verkannten die Vorinstanzen den Fehlerbegriff des PrHG. Es argumentierte, die Geschädigte habe nicht die Ursache des Mangels zu beweisen, sondern es genüge, wenn sie aufzeige, dass das Produkt die berechtigten Sicherheitserwartungen des durchschnittlichen Konsumenten nicht erfülle. 

Da es sich um eine Kausalhaftpflicht handelt, kann sich der Hersteller und die ihm gleichgestellten Inverkehrbringer nicht damit exkulpieren, sie hätten die gebotene Sorgfalt angewendet. 
Komme es im Zusammenhang mit dem Gebrauch eines Produktes zu einem Unfall, so das Bundesgericht, beurteile sich der Beweis des Geschehensablaufs, der zum Unfall geführt hat, im Prinzip nach dem Gesichtspunkt der überwiegenden Wahrscheinlichkeit.
Im Ergebnis bedeutet dies eine Erleichterung für die Klägerin. So kann sie einfach argumentieren, die Produktsicherheit habe nicht dem von der Allgemeinheit berechtigterweise erwarteten Standard
entsprochen.
