\chapter{IT Verträge}

\begin{description}
  \item[Abschluss des Vertrags] \textit{OR 1}
  \item[Wesentliche Punkte zur Einigung] \textit{OR 2} 
  \item[Formvorschrift] \textit{OR 11} 
  \item[Inhalt des Vertrags] \textit{OR 19} 
\end{description}

\section{Grundlagen}

IT-Verträge gehören zum betrieblichen Alltag. Sie zählen zu den Innominatverträgen und enthalten oftmals Elemente des Auftrags und/oder des Werkvertrages oder auch von anderen Vertragstypen, wie bspw. der Miete, Hinterlegung, Pacht oder dem Kauf.

Die Abgrenzung ist oft schwierig und muss einzelfallspezifisch vorgenommen werden. Umso wichtiger ist es, im Vorfeld den Vertrag genau auf die Absichten der Parteien hin zu formulieren und auszulegen.

Die Bestimmungen des OR kommen analog zur Anwendung. Umso wichtiger ist es, die einzelnen Elemente des Vertrages richtig zu subsumieren (Mängelrechte, Verjährung, Haftungsausschlüsse, Rügeobliegenheit etc.).
\begin{itemize}
  \item Entwicklung einer App, Erstellung einer Website oder Entwicklung einer Individualsoftware, Installations- und Reparaturarbeiten = Erfolg ist i.d.R. geschuldet, daher überwiegend Werkvertrag.
  \item Nutzungsrechte bei Cloudverträgen = Können Elemente der Miete enthalten.
  \item Selbstständige Planung, Beratung und Projektmanagementleistungen = Das sorgfältige Tätigwerden ist im Vordergrund, daher überwiegend Auftrag.
  \item Erwerb einer Standardsoftware: Die Regelungen betreffend den Kaufvertrag kommen analog zur   Anwendung. Individuell konzipierte IT-Produkte haben aber werkvertraglichen Charakter.
  \item Wo die Bestimmungen des OR BT nicht oder nur beschränkt angewendet werden können, kommen die   OR AT-Bestimmungen (z.B. Folgen der Nichterfüllung OR 97ff. oder Verjährung OR 127ff.) zum Tragen.
\end{itemize}

\section{Beispiele wichtiger IT Verträge}

\begin{description}
  \item[Software-Lizenzvertrag] Einräumung von Rechten zum Gebrauch von Applikationen während längerer Dauer, gegen die Entrichtung von Benützungsgebühren. Bspw. unbefristete Lizenzierung, Abonnement-basierte Lizenzierung (Spotify, Adobe, Netflix), Netzwerklizenzen, Cloudbasierte Lizenzierung
  \item[Wartungsvertrag] Vertragsgegenstand ist der Erhalt/Verbesserung und/oder die Wiederherstellung der Betriebsbereitschaft der Software oder die Aktualisierung, Beratung sowie Pflege der Software.
  \item[Dienstleistungsvertrag] für Planung und Beratung  
  \item[Lieferung von integrierten Informatiksystemen]
  \item[Systemintegrationsvertrag] Planung, Einrichtung und Aufbau von IT-Systemen. Prüfung von Kompatibilität von Hardware- und Softwarekomponenten
  \item[IT-Werkvertrag] Es ist ein bestimmbares oder bestimmtes Ergebnis geschuldet. Das Werk muss genau, auch für Dritte nachvollziehbar, umschrieben und definiert werden.
  \item[IT-Dienstleistungsvertrag] Der Beauftragte verpflichtet sich, für den Kunden in fachgerechter Sorgfalt tätig zu werden, nicht aber zur Realisierung eines bestimmten Erfolges.
  \item[Outsourcingvertrag] Outside Resource Using, Auslagerung von IT-Systemen oder Prozessen
  \item[Software Escrow Agreement] Entwickler hinterlegt den Source Code von Software bei einem neutralen Dritten. Der Escrow-Agent bewahrt den Source Code sicher auf. Erst wenn eine der vordefinierten Voraussetzungen eintritt, gibt der Escrow-Agent dem Kunden den Source Code heraus.
  \item[Cloud Services] IaaS, SaaS, PaaS
  \item[Hardwareverkaufsvertrag]
  \item[Softwareentwicklungsvertrag] Entwickler erstellt eine Software-Applikation nach den konkreten Vorgaben des Bestellers
  \item[Roamingverträge] Nutzung des mobilen Endgerätes in einem ausländischen Netz für Telefonate, Nachrichten und mobile Daten
  \item[Software-Distributionsvertrag] Hersteller oder Lieferant beauftragt einen Vertreiber mit dem Verkauf seiner Produkte
  \item[Erbringung von Hosting-Dienstleistungen] Überlassung von Speicherplatz auf der Serverinfrastruktur des Anbieters für die Website oder Applikation des Kunden sowie die Erbringung dazugehöriger Dienstleistungen.
  \item[Konzeption und Realisierung einer Web-Applikation] Softwareprogramm, das auf einem Webserver ausgeführt wird. Im Gegensatz zu Desktop-Anwendungen, die lokal auf einem Computer installiert werden, muss auf Webanwendungen über einen Webbrowser zugegriffen werden. Bspw. Webshop oder Internet-Banking-Programme, Evernote, Google Apps, Pocket.
  \item[Erstellen einer Webseite]   
\end{description}

\section{Leistungsstörungen}

\begin{itemize}
  \item Das IT-System funktioniert nicht wie erwartet/zugesichert
  \item Softwarefehler führen zu Systemausfällen und Datenverlusten
  \item Anfälligkeit des Systems auf Hackerangriffe
  \item Systemzugriffe sind nicht möglich
  \item Prozesse laufen nicht wie vorgesehen ab
\end{itemize}

\noindent
Mögliche Konsequenzen
\begin{itemize}
  \item Produktionsausfälle (Einnahmenverlust, Schadenersatzansprüche von Vertragspartnern, weitere mittelbare Schäden)
  \item Personenschäden bei Systemausfällen/Fehlsteuerung von sicherheitsrelevanter Software
  \item Reputationsschäden
  \item Persönlichkeits- und Datenschutzverletzungen
\end{itemize}

\noindent
Mögliche Haftungsansprüche
\begin{itemize}
  \item Ausservertragliche Haftungsansprüche (z.B. Delikt- (OR 41), Geschäftsherren- (OR 55) und Produktehaftung (PrHG))
  \item Vertragliche Haftungsansprüche (z.B. Arbeitsrechtliche Ansprüche (OR 321e), Gewährleistungs- und Haftungsansprüche (z.B.
OR 97))
  \item Verantwortlichkeitsansprüche
\end{itemize}

\subsection{Mängelrechte}

Bei Mängeln ist jeweils darauf zu achten, welche rechtlichen Grundlagen am ehesten zur Anwendung kommen, sofern die Parteien diesbezüglich nichts vereinbart haben:

Mängel und Verzug bei:
\begin{description}
  \item[kaufvertraglichen] Eigenschaften s. Art. 97ff. OR und die besonderen
  Bestimmungen zum Kaufvertrag insb. Art. 197ff. OR (Wandelung, Minderung
  oder Ersatzlieferung)
  \item[werkvertraglichen] Eigenschaften s. Art. 97ff. OR und die besonderen
  Bestimmungen zum Werkvertrag Art. 363ff. OR (Wandelung, Minderung oder
  Nachbesserung)
  \item[auftragsrechtlichen] Eigenschaften s. Art. 97ff. OR und die besonderen
  Bestimmungen zum Auftrag Art. 394ff. OR (Schadenersatz)
\end{description}

\section{Risiken / Probleme}

\begin{itemize}
  \item Uneinigkeit betreffend der zu erbringenden Leistung
  \item Mangelhafte Vertragserfüllung (Nichterfüllung/Schlechterfüllung)
  \item Haftung
  \item Datenschutzverletzungen
  \item Sich verändernde Bedürfnisse des Leistungsnachfragers
  \item Höhere Kosten als erwartet/veranschlagt
  \item Ausstehende Zahlungen
  \item Vorzeitige Vertragsauflösung
\end{itemize}

\section{Mögliche wichtige Regelungspunkte in den Verträgen}

\begin{description}
  \item[Regelungen den Datenschutz und die Informationssicherheit betreffend]Werden bspw.Personendaten bearbeitet? Welchen Sicherheitsstandard hat der Provider?
  \item[Geheimhaltung] Der Dienstleistungserbringer hat nicht selten Einblick in wichtige und geheim zu haltenden Geschäftsinformationen.
  \item[Gewährleistungsrecht und Haftung] Für welche Leistung muss wer, wie einstehen?

  Als Mangel wird in der Technik generell eine Abweichung von Vorgaben angesehen, während im rechtlichen Sinne ein Mangel ursächlich für eine Beeinträchtigung der vertraglich geschuldeten Leistung (subjektive- und objektive Wesentlichkeit) sein muss.
  Mängel können sein: Schlechte Kostenschätzung oder Kostenkontrolle, mangelnde   Benutzerfreundlichkeit, rechtliche Mängel (fehlende Lizenzen), notwendige Funktionen fehlen, Viren in der Software, keine ausreichende Performance, Drittkomponenten sind nicht kompatibel oder ganz allgemein, das Fehlen von zugesicherten Eigenschaften
  
  Sind die Mangelanmeldeprozesse eindeutig definiert? Ist definiert, was Mangelbehebung und was ein
  kostenpflichtiger Support ist? Wer haftet und wie (beschränkt oder unbeschränkt)?
  
  \item[Vertragsauflösung] Ordentliche- und ausserordentliche Auflösung
\end{description}

\subsubsection*{Gewährleistung und Abnahme der Leistungen}
\begin{enumerate}
  \item Gewährleistung
  \begin{enumerate}
    \item Grundsatz: Optiwork erbringt ihre Leistungen mit grösster Sorgfalt und nach bestem Wissen und Gewissen. Sie gewährleistet das zwischen den Parteien vereinbarte Funktionieren der Produkte und Dienstleistungen. Kleinere, die Tauglichkeit zum vorausgesetzten Gebrauch nicht erheblich mindernde oder aufhebende Fehler können allerdings nicht mit Sicherheit vermieden werden.
  \item Für eigene Software: Es gelten die Bestimmungen des Softwarelizenzvertrages DOMUS.
  \item Für Hardware und Fremdsoftware: Es gelten die Garantiebestimmungen des Drittlieferanten.
  \end{enumerate}
  \item Abnahme: Die Abnahme erfolgt nach erbrachter Leistung. Ein Abnahmeprotokoll ist nur aufgrund einer speziellen, gegenseitigen Vereinbarung vorgesehen.
  \item Nach erfolgter Abnahme: Für die Zeit nach der Abnahme schliessen die Vertragsparteien einen Wartungsvertrag ab.
  \item Mängelrüge: Eine detaillierte Mängelrüge ist vom Kunden schriftlich während eines laufenden Projektes oder spätestens 30 Tage nach Erhalt einer Teil- oder Schlussrechnung an Optiwork zuzustellen.
  Als Mängel verstehen die Vertragsparteien wesentliche Abweichungen von der Spezifikation gemäss Auftragsbestätigung, Vertrag oder des Programmbeschriebes, die den Wert oder die Tauglichkeit für den im Vertrag oder Auftrag vorgesehenen Gebrauch aufheben oder erheblich mindern.
  \item Kostenlose Mängelbehebung: Optiwork behebt gerügte und von Optiwork anerkannte Mängel ohne Kostenfolge für den Kunden. Aufwendungen der Kunden werden nicht von BRZ übernommen.
  \item Kostenpflichtige Mängelbehebung: Stellt sich nach gemeinsamer Prüfung durch beide Parteien heraus, dass ein gerügter Mangel nachweislich nicht durch Optiwork zu vertreten ist, so ist der Kunde verpflichtet, der Optiwork die entstandenen Aufwendungen nach den jeweils gültigen Preisen und Bedingungen zu vergüten.
  \item Weitergehende Ansprüche: Andere Rechtsbehelfe als die Nachbesserung werden, vorbehältlich von Schadenersatz im Falle von grober Fahrlässigkeit oder Vorsatz, ausdrücklich ausgeschlossen. Insbesondere ist der Kunde nicht berechtigt, die Nachbesserung selber vorzunehmen oder durch Dritte vornehmen zu lassen und die entsprechenden Kosten gegenüber Optiwork geltend zu machen
\end{enumerate}

\subsubsection*{Haftung}

\begin{enumerate}
  \item Voraussetzungen: Bei Vertragsverletzungen oder unerlaubten Handlungen, welche von Optiwork oder deren Mitarbeitenden zu verantworten sind, haftet Optiwork nur, soweit die Vertragsverletzung oder eine unerlaubte Handlung durch rechtswidrige Absicht oder grobe Fahrlässigkeit verursacht wurde.
  \item Begrenzung der Haftung: Sofern in den einzelnen Verträgen keine weitergehende Begrenzung der Haftung und der Schadenersatzpflicht vereinbart ist, wird die Höchstsumme des Schadenersatzes in jedem Fall auf die seitens des Kunden unter dem   entsprechenden Einzelvertrag bereits geleisteten jährlichen Zahlungen oder geschuldeten Entgelte beschränkt. Dies gilt auch für Schadenersatzansprüche des Kunden im Fall der Kündigung des Vertrages oder des Rücktrittes vom Vertrag durch den Kunden.
  \item Folgeschäden: Optiwork haftet auf keinen Fall für Mangelfolgeschäden, mittelbare Schäden, entgangenen Gewinn, Frustrationsschäden oder Ansprüche Dritter, die an den Kunden gestellt werden.
  \item Ausschluss von Gewährleistungs- und Haftungsansprüchen: Sämtliche Haftungs- und Gewährleistungsansprüche werden ausgeschlossen, wenn Dritte ohne ausdrückliche Einwilligung von Optiwork Eingriffe an deren Produkten vornehmen oder wenn der Kunde   ohne ausdrückliche Zustimmung von Optiwork Änderungen an der Hardware-Charakteristika, Installationen, Betriebssoftware oder anderer mit der Leistung von Optiwork im Zusammenhang stehenden Software vornimmt oder vornehmen lässt.
  \item Unzureichende Mitwirkung des Kunden: Wenn der Mangel auf der vom Kunden gegebenen Aufgabenstellung oder der fehlerhaften und/oder unzureichenden Mitwirkung des Kunden beruht, werden alle Haftungs- und Gewährleistungsansprüche ausgeschlossen.
  \item Verhinderung der Leistungserbringung: Wird Optiwork aus Gründen, welche sie nicht zu vertreten hat, an der zeitgerechten oder sachgemässen Erfüllung der Verpflichtungen aus dem Vertrag gehindert, so entsteht ebenfalls kein Haftungsanspruch.
\end{enumerate}

\section{Mängelrechte}

Bei Mängeln ist jeweils darauf zu achten, welche rechtlichen Grundlagen am ehesten zur
Anwendung kommen, z.B.:

\begin{itemize}
  \item Bei Mängeln und Verzug bei kaufvertraglichen Eigenschaften s. Art. 97ff. OR und die besonderen Bestimmungen zum Kaufvertrag insb. Art. 197ff. OR (Wandelung, Minderung oder Ersatzlieferung)
  \item Bei Mängeln und Verzug bei werkvertraglichen Eigenschaften s. Art. 97ff. OR und die besonderen Bestimmungen zum Werkvertrag Art. 363ff. OR (Wandelung, Minderung oder Nachbesserung)
  \item Bei Mängeln und Verzug bei auftragsrechtlichen Eigenschaften s. Art. 97ff. OR und die besonderen Bestimmungen zum Auftrag Art. 394ff. OR (Schadenersatz)
\end{itemize}

\section{Outsourcing-Vertrag}

Das Leistungsspektrum kann sehr breit und verschiedenartig sein. Als Folge davon kann der Outsourcing- Vertrag oftmals nicht unter einen der gesetzlich geregelten Vertragstypen subsumiert werden. Es handelt sich bei diesem Vertrag um einen Innominatvertrag. Er stellt nicht selten einen gemischten Vertrag dar, der Elemente verschiedener Vertragstypen enthält (Bspw. Miete, Pacht, Werkvertrag oder Auftrag).

\subsection{Erscheinungsformen}

Der Outsourcingvertrag kann unterschiedlich aufgebaut sein.
Es kommt vor, dass \textbf{ein bestimmter Vertrag} für eine oder mehrere Dienstleistungen
ausgefertigt wird.
Oder es wird ein \textbf{Rahmenvertrag} abgeschlossen, der für eine Mehrzahl von Dienstleistungen die grundlegenden vertraglichen Bestimmungen festlegt. Aufbauend auf diesem Vertrag werden für jede einzelne Dienstleistung sogenannte Einzelverträge abgeschlossen, die die Spezifika der betreffenden Dienstleistung festlegen.
Ergänzt werden solche Rahmenverträge und Einzelverträge oftmals durch \textbf{Service Level Agreements (SLAs)}, welche genauer die Leistungsmerkmale für gewisse Dienstleistungen definieren. 

SLAs dienen der detaillierten Beschreibung von Art, Quantität und Qualität vereinbarter Outsourcing-Services, wodurch Dienstleistungen messbar gemacht werden. Während technische und operative Bestimmungen auf Stufe SLA festzuschreiben sind, müssen die \textbf{zentralen strategischen und kommerziellen Punkte} im Outsourcing-Vertrag selbst Berücksichtigung finden. Andernfalls besteht die Gefahr, dass durch nachträgliche Ergänzung oder Änderung von SLAs der Sinn eines Outsourcing-Vertrages ausgehöhlt wird.
\vspace{3mm}

Beim Outsourcing-Vertrag ist die Frage der \textbf{Vertragsqualifikation} relevant. Käme Auftragsrecht zum Tragen, wäre der Vertrag jederzeit kündbar, das kann aber nicht im Sinne der Parteien sein.
Vielmehr dürften aufgrund der konkret feststell- und messbaren Leistungspflicht des Dienstleistungserbringers eher werkvertragsähnliche Elemente überwiegen. Werden dem Kunden auch Systemkapazität zur Verfügung gestellt, hat der Outsourcing-Vertrag auch mietvertragsähnlichen Elemente.

Untergeordnete auftragsrechtliche Elemente können bei reinen Beratungsleistungen oder bei einem Tätigwerden für den Kunden, bei welchem keine konkret messbare Leistungspflicht verbunden ist, auftreten. 

Wichtige Phasen des Outsourcings sind:
\begin{itemize}
  \item die Auslagerung der IT-Infrastruktur
  \item der anschliessende Betrieb
  \item und die Rückübertragung
  \item Über diese Phasen sollte der Vertrag Auskunft geben können.
\end{itemize}

\vspace{3mm}

\subsection{Beispiele}

Das Outsourcing kann darauf beschränkt sein, eine bestimmte IT-Infrastruktur zur Verfügung zu stellen, auf der der Nutzer seine Aufgaben selbst vollzieht (\textbf{Systems Management}).
Beim \textbf{Application Management} übernimmt der Outsourcing-Provider die Aufgabe, eine Softwareanwendung zu betreuen.
Der Outsourcing-Provider kann neben dem Betrieb der IT-Infrastruktur weitere Funktionen wie die Abwicklung von Geschäftsvorgängen des Outsourcing-Bezügers übernehmen. Dann spricht man von einem \textbf{Business Process Outsourcing}.

Beispiele zur vertragstypologischen Einordnung:
Bei der \textbf{Entwicklung und Einführung einer neuen IT-Systemumgebung}, unterliegt die vertragliche Beziehung primär dem \textbf{Werkvertragsrecht}. Hinzu können auftragsrechtliche Elemente treten.

Bei der \textbf{Übernahme einer bestehenden IT-Infrastruktur} stehen \textbf{kauf-, miet- und lizenzvertragliche Elemente} im Vordergrund, wobei die unterschiedlichen Bestimmungen betreffend die Übertragung von Infrastrukturkomponenten jeweils besonders zu berücksichtigen sind.

Werden auch \textbf{Personalressourcen} ausgelagert, sind zudem \textbf{arbeitsrechtliche Bestimmungen} massgeblich.

Der eigentliche \textbf{Betrieb der ausgelagerten Systeme, Anwendungen oder Prozesse} hat primär \textbf{werkvertraglichen Charakter}, falls vom Outsourcing-Provider eine Erfolgsgarantie übernommen wird, was beim Einsatz von Service Level Agreements die Regel darstellt.

Der Dauerschuldcharakter des IT-Outsourcing bildet dabei ein atypisches Vertragselement. Beschränkt sich das Outsourcing auf das \textbf{reine Bereithalten von Rechnerkapazität ohne Erbringung zusätzlicher Datenverarbeitungsleistungen}, rückt das Outsourcing in die Nähe eines \textbf{Miet- oder Pachtvertrages}. Oft können auch Merkmale des \textbf{Auftrages (z.B. Schulung/Support)} oder des \textbf{Hinterlegungsvertrages (bei der Datenauslagerung)} vorgefunden werden.

\subsection{Wichtige Regelungspunkte}
\begin{description}
  \item[Messung des Leistungsbezugs] Fälligkeitstermine, Abrechnungs‐ und Zahlungsmodalitäten, Verzug und Lösung von Differenzen über Zahlungsverpflichtungen
  \item[Informationssicherheit und Datenschutz]
  \item[Gewährleistung und Haftung]
  Beim IT-Outsourcing treten überwiegend mittelbare Vermögensschäden (z.B. als Folge von Betriebsunterbrüchen oder in Form entgangenen Gewinns) auf, deren Nachweis und Quantifizierung naturgemäss schwer fällt und für welche die Haftung meist ausgeschlossen oder beschränkt wird.

  Deshalb werden in Outsourcing-Verträgen meist alternative Formen der geldwerten Ersatzleistung vorgesehen, wobei zwei Erscheinungsformen im Vordergrund stehen:

  Zunächst kann für den Fall der Nichterreichung von Service Levels eine Konventionalstrafe im Sinne von Art. 160 ff. OR vereinbart werden. Die Abrede einer Konventionalstrafe hat für den Outsourcing-Bezüger den Vorteil, dass kein konkret entstandener Schaden nachgewiesen werden muss. Aus Sicht des Providers erscheint eine Konventionalstrafe aber häufig als einseitig und undifferenziert.

  Stattdessen können anstelle von Konventionalstrafen Bonus-Malus-Systeme eingesetzt werden, bei denen nach dem Grad der Zielerreichung abgestufte Gutschriften bzw. Belastungen ausgelöst und meist über die Vergütung abgerechnet werden.
  \item[Vertragsdauer und Beendigung] Ordentliche oder ausserordentliche Kündigung, Folgen der Vertragsauflösung (Backsourcing).
  \item[Weitere Bestimmungen] Anwendbares Recht und Gerichtsstand, Rechntsnachfolge, Formerfordernisse bei Anpassungen und Unterschriften
\end{description}

\subsection{Vorteile}

\begin{itemize}
  \item Kosteneinsparung (Anbieter stehen im Wettbewerb und Know-how der Anbieter kann genutzt werden)
  \item Effizienzgewinn (Anbieter ist daran interessiert, die optimale Lösung anzubieten und es ist sein Kerngeschäft)
  \item klare Kostenstruktur
  \item Straffung der eigenen Organisation
  \item Fokus auf eigenes Kerngeschäft
\end{itemize}

\subsection{Nachteile}

\begin{itemize}
  \item Bei einem Outsourcing gliedert man oftmals betriebskritische Systeme und/oder Prozesse auf einen Dritten aus und ist davon abhängig, dass der Outsourcing-Provider über einen längeren Zeitraum die vereinbarten Dienstleistungen zu den vereinbarten Service Levels und Preisen erbringt.
  \item Leistungsstörungen während des Betriebes können sich gravierend auf das Geschäft des Kunden auswirken.
  \item Nutzer gibt einen Teil der Kontrolle über seine Daten ab.
\end{itemize}