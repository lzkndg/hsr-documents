\chapter{Verträge auf Arbeitsleistung}

\section{Werkvertrag}

\textit{363 OR}: Durch den Werkvertrag verpflichtet sich der Unternehmer zur Herstellung eines Werkes und der Besteller zur Leistung einer Vergütung.

Merkmale
\begin{itemize}
  \item Im Zentrum der Leistung steht das Werk
  \item Der \textbf{Erfolg ist geschuldet}.
  \item Unternehmer verwendet i.d.R eigene Arbeitsmittel
  \item Einmalige Leistungserbringung
\end{itemize}

Der Werkvertrag kann formfrei geschlossen werden. 

\textit{Beispiele:} 
Tuning, Herstellung eines Eherings, Haarschnitt, Hausbau, Anfertigen eines Möbelstücks, Bauplan des Ingenieurs, Anzug nach Mass, Autoreparatur, Aufbau einer umfassenden und "massgeschneiderten" IT- Landschaft, aber auch Innominatverträge können Eigenschaften eines Werkvertrages aufweisen (Gemischter Vertrag: Kauf einer Maschine mit diversen Aufbau- und Programmierarbeiten)

\subsection{Vergütung}

Fixpreis nach \textit{OR 373 Abs. 1}

\textit{1 Wurde die Vergütung zum voraus genau bestimmt, so ist der Unternehmer verpflichtet, das Werk um diese Summe fertigzustellen, und darf keine Erhöhung fordern, selbst wenn er mehr Arbeit oder grössere Auslagen gehabt hat, als vorgesehen war.}
\vspace{3mm}

\noindent
Keine oder ungefähre Preisvereinbarung nach \textit{OR 374}

\textit{1 Ist der Preis zum voraus entweder gar nicht oder nur ungefähr bestimmt worden, so wird er nach Massgabe des Wertes der Arbeit und der Aufwendungen des Unternehmers festgesetzt.}
\vspace{3mm}

\noindent
Überschreitung des Kostenansatzes nach \textit{OR 375}

\textit{1 Wird ein mit dem Unternehmer verabredeter ungefährer Ansatz ohne Zutun des Bestellers unverhältnismässig überschritten, so hat dieser sowohl während als nach der Ausführung des Werkes das Recht, vom Vertrag zurückzutreten.}

\textit{2 Bei Bauten, die auf Grund und Boden des Bestellers errichtet werden, kann dieser eine angemessene Herabsetzung des Lohnes verlangen oder, wenn die Baute noch nicht vollendet ist, gegen billigen Ersatz der bereits ausgeführten Arbeiten dem Unternehmer die Fortführung entziehen und vom Vertrage zurücktreten.}

Vergütung kann \textbf{nach Norm} (bsp. SIA) gesondert geregelt werden. Einheitspreis, Globalpreis, Regiepreis, Pauschalpreis, \ldots

\subsection{Folgen bei Vertragsverletzung}

\begin{description}
  \item[Mängelhaftung (OR 367ff)] \textit{(zur Beurteilung des Verschuldens OR 364 Abs. 1)}
  Voraussetzung zur Ausübung der Mängelrechte
  \begin{itemize}
    \item Werk ist bei Ablieferung mangelhaft.
    \item Besteller erfüllt Prüf- und Rügeobliegenheit: Stillschweigende bzw. ausdrückliche Genehmigung des Werkes \textit{(OR 370 Abs. 1)}.
    \item Rechtzeitige Rüge gemäss \textit{OR 367 Abs. 1}: nach dem üblichen Geschäftsgange tunlich. \textit{Versteckte Mängel} müssen sofort nach ihrem Entdecken gerügt werden. 
    \item Keiner der folgenden Ausschlussgründen liegt vor: Mangel wird durch den vom Besteller gelieferten Stoff / Baugrund verursacht und der Unternehmer zeigt dies an, Der Besteller erteilt unsachgemässe Weisungen zur Ausführung des Werkes (OR 369), Gewährleistung wurde nicht wegbedungen (OR 100/199).
    \item Der Anspruch ist nach \textit{OR 371} nicht verjährt.
  \end{itemize}

  Folgen bei Mängeln:
  \begin{itemize}
    \item Wandelung (OR 386 Abs. 1)
    \item Minderung (OR 368 Abs. 2)
    \item Nachbesserung (OR 368 Abs. 2)
    \item Schadenersatz (OR 369 Abs. 1 und 2)
  \end{itemize}

  \item[Bei Verzug (OR 366, OR 102 ff)]
  Befindet sich der Unternehmer im Verzug, so kann der Besteller vorzeitig vom Vertrag zurücktreten. Falls die Fertigstellung des Werkes unmöglich wird gilt \textit{OR 97ff}.
  \textbf{Eine Ersatzvornahme} kann unter bestimmten Umständen (OR 366 Abs. 2) durchgeführt werden.
  \begin{itemize}
    \item Voraussehbarkeit der mangelhaften oder sonst vertragswidrigen Herstellung des Werkes (Prognoseentscheidung)
    \item Verschulden des Werkunternehmers (kein Selbstverschulden des Bestellers)
    \item Angemessene Fristansetzung mit Androhung für den Fall der Unterlassung (Ersatzvornahme)
  \end{itemize}
  Sind diese Voraussetzungen erfüllt, kann der Besteller alternativ auch vom Vertrag zurücktreten \textit{(OR 107 Abs. 2)}.
    
  \item[Betreffend den Stoff] \textit{OR 365} Sachgewährleistung wenn der Unternehmer den Stoff liefert, Sorgfaltspflicht wenn der Besteller den Stoff liefert. Orientierungspflicht des Unternehmers falls Mängel am Material ersichtlich werden.
  \item[Vorzeitiger Rücktritt] ist jederzeit möglich, gegen volle Schadloshaltung des Unternehmers (Vergütung der bereits geleisteten Arbeit und getätigten Auslagen).
\end{description}

\section{Auftrag}

Im OR sind verschiedene Varianten eines Auftrags geregelt:
\begin{itemize}
  \item Einfacher Auftrag \textit{(OR 394 ff)}
  \item Auftrag zur Ehe- oder Partnerschaftsvermittlung \textit{(OR 406a)}
  \item Der Kreditbrief und der Kreditauftrag (verpflichtet den Beauftragten, einem Dritten Kredit zu gewähren)
  \item Der Mäklervertrag \textit{(OR 412 Abs. 1)}
  \item Der Agenturvertrag \textit{(OR 418a)}
\end{itemize}


\begin{description}
  \item[Merkmale] eines Auftrags:
  \begin{itemize}
    \item Im Zentrum steht das \textbf{sorgfältige Tätigwerden}
    \item Der \textbf{Erfolg ist nicht geschuldet}
    \item \textbf{Treueverpflichtung} des Beauftragten
    \item Besonderes Vertrauensverhältnis
    \item Selbständige Stellung des Beauftragten
  \end{itemize}

  \item[Gegenstand] Kann formfrei geschlossen werden. Im Grundsatz hat der Beauftragte den Auftrag vertragsgemäss auszuführen. Zwar ist der Erfolg nicht geschuldet. Der Beauftragte ist aber verpflichtet einen solchen anzustreben und in diesem Sinn tätig zu werden.
  Gegenstand eines Auftrages kann jede beliebige persönliche Handlung sein. Voraussetzung für das Vorliegen eines Auftrages ist stets, dass es sich um ein Tätigwerden in fremdem Interesse handelt.

  Die dem Beauftragten übertragenen Geschäfte können Rechtshandlungsaufträge oder auch Tathandlungsaufträge sein. Auch ist der Auftrag ein Sammelbecken für alle Arbeitsleistungs- und Dienstleistungsverträge, welche nicht einem gesetzlichen Sondertypus entsprechen.

  Bspw.: Medizinische Behandlung, anwaltliche Beratung, Termin bei einem Steuerberater, Finanzberatung oder Unternehmensberatung

  \item[Sorgfaltspflicht] \textit{1 Der Beauftragte haftet im Allgemeinen für die gleiche Sorgfalt wie der Arbeitnehmer im Arbeitsverhältnis.} 
  \textit{2 Er haftet dem Auftraggeber für getreue und sorgfältige Ausführung des ihm übertragenen Geschäftes.}

  Zwar wird die gleiche Sorgfalt wie von einem Arbeitnehmer erwartet. Die entsprechenden arbeitsrechtlichen Bestimmungen dürfen aber nicht 1:1 übernommen werden. Ob der Beauftragte unsorgfältig gehandelt hat, beurteilt sich danach, ob ihm sein Handeln unter Berücksichtigung der konkreten Umstände gemessen am fachspezifischen Durchschnittsverhalten vorwerfbar ist.

  Allgemein besteht ein Verschulden des Beauftragten darin, dass er die Vertragsverletzung vorsätzlich oder fahrlässig herbeigeführt hat. Fahrlässig handelt der Beauftragte, wenn er die im Verkehr erforderliche Sorgfalt verletzt.
  \item[Treuepflicht] \textit{(OR 400)}
  \textit{Der Beauftragte ist schuldig, auf Verlangen jederzeit über seine Geschäftsführung Rechenschaft abzulegen und alles, was ihm infolge derselben aus irgendeinem Grunde zugekommen ist, zu erstatten.}
  Der Beauftragte muss die Interessen seines Auftraggebers wahren. Der Beauftragte ist demnach bspw. verpflichtet, den Auftraggeber zu beraten, kritisch zu hinterfragen, informieren, Informationen geheim zu halten.
  Auch soll der Auftraggeber die Fähigkeiten und persönlichen Eigenschaften besitzen, um den Auftrag nach bestem Wissen und Gewissen erfüllen zu können.
  \item[Widerruf / Kündigung] \textit{(OR 404, nicht dispositiv!)}
  \textit{Der Auftrag kann von jedem Teile jederzeit widerrufen oder gekündigt werden.} 
  
  \textit{2 Erfolgt dies jedoch zur Unzeit, so ist der zurücktretende Teil zum Ersatze des dem anderen verursachten Schadens verpflichtet.}

  Eine Vertragsauflösung zur Unzeit liegt gemäss bundesgerichtlicher Rechtsprechung vor, wenn der Zeitpunkt der Kündigung besonders ungünstig ist und für den Vertragspartner besondere Nachteile mit sich bringt.

  Beispiel: Angemeldete Weiterbildung konnte aus gesundheitlichen Gründen nicht angetreten werden. Zahlung schon erfolgt, wurde nicht zurückerstattet. Nach Abklärung: Kurs war ausgebucht, die Schule hatte also keinen Finanziellen Schaden. Geld musste zurückerstattet werden.
\end{description}  