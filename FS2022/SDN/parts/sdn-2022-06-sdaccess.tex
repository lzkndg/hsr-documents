\section{Software Defined Access}

The term ``Fabric" describes an overlay. This is a logical topology, which virtually connects specific 
devices using some unspecified underlaying infrastructure.
Control planes for underlay and overlay are completely separate.

Overlay transported traffic usually only ``knows about'' provider edge (PE) routers. 
Traffic routing inside the underlay network is handled by the overlay logic and also the underlay routing protocol.

\vspace{3mm}
\noindent
Overlay can transport layer 2 or 3 traffic , using some kind of encapsulation:
\begin{itemize}
    \item Layer 2 emulates a LAN segment, can provide different physical topologies. \\
    Supports: Layer 2 flooding, single subnet mobility, transports ethernet frames.
    \item Layer 3 abstracts IP conncectivity. \\
    Supports: Gateway-independent mobility, contain network failures, transport IP packets (IPv4/IPv6)
\end{itemize}

\noindent
Underlay usually uses some IGP for connectivity and control plane tasks.
\begin{itemize}
    \item OSPF
    \item IS IS
    \item Manual or automated deployment
    \item Low latency is important (max. 100ms)
\end{itemize}

\subsection{VXLAN}

VXLAN encapsulates layer 2 frames in UDP/IP, emulating a dedicated LAN network on top of a routed layer 3 network.

UDP Destination is port 4789, source port is chosen from a range of high ports to allow Equal Cost Multi-Pathing.

24-bit VXLAN segments allow for more than 16 million VNIDs, while classic VLANs are limited to 12 bit (4096 VLAN IDs).

Encapsulation of MAC in UDP can be done in hardware / at line rate.

Local Station Table - LST

Global Station Table - GST

Unknown destinations are forwarded to a spine -$>$ Council of Oracle. The oracle (a spine node) knows everything. Because everything learned by a leaf is forwarded to the spine.

\subsection{LISP}
\emph{Locator Identifier Separation Protocol 
(\href{https://de.wikipedia.org/wiki/Locator/Identifier_Separation_Protocol}{Wikipedia})}

\vspace{3mm}
\noindent
Separates \textbf{EID, endpoint identities} and \textbf{RLOC, routing locators}, which an IP address usually implicitly both contains, and maps one to the other.

\begin{itemize}
    \item Map Server / Resolver \\
        Control plane function 
    \item Tunnel Router \\
        Ingress ITR, Egress ETR \\
        Registers EID of clients with Map Server
        encapsulates and decapsulates traffic of edge devices
    \item Proxy Tunnel Router PxTR \\
        Ingress (PITR) connects LISP- and non-LISP-domains \\
        Egress PETR
    \item Solicit Map Request \\
        is sent when a client location is unknown
\end{itemize}

\noindent
\emph{Control Plane Register}
\begin{enumerate}
    \item Router communicates Map-Register to controller (EID, RLOC)
    \item Controller builds its database
\end{enumerate}

\vspace{3mm}
\noindent
\emph{Map Request (Proxy Mode and Non-Proxy Mode)}
\begin{enumerate}
    \item Map-request for requesting a location information from the controller. 
    \item Non-Proxy: The request is then delegated to and will be answered by the corresponding end-router.
    \item Proxy: Another request will be made to the correct router, answer will be sent by controller.
    \item Request answer will be cached by the requester.
\end{enumerate}

\vspace{3mm}
\noindent
\emph{External Networks known/unknown}
\begin{enumerate}
    \item Known external networks can be listed inside the database exactly like internal hosts.
    \item Unknown networks are processed as a ``default route'' / ``not found'' entry with separate destination / location.
\end{enumerate}

\vspace{3mm}
\noindent
\emph{Host Mobility}
\begin{enumerate}
    \item Known device connects with a different XTR: map entry is created.
    \item Requests based on old, cached information by an ITR will be answered with ``Solicit Map-Request'' by the ETR that does not know the client anymore.
    \item Standard Map-Request can then be sent to the correct ETR, and cache will be updated 
\end{enumerate}


\subsection{Secure Group Tag}
Topology-independent, role based access control.

Scaleable ingress tagging (SGT), egress filtering (SGACL).

Centralized policy management on DNA controller, distributed policy enforcement on all network devices.

Secure Group Tag introduces a micro level of segmentation, while VLAN and VXLAN operate on a macro level. 

\subsection{Cisco DNA Center}
Acts as Management Controller of a campus fabric. Integrates several management systems to configure and orchestrate LAN, WLAN and WAN access.

\begin{itemize}
    \item Fabric Control Plane
        \begin{itemize}
            \item Different Endpoint ID Lookups are supported (IPv4, IPv6, MAC)
            \item Maintanis Endpoint ID Map Registrations
            \item Processes Lookup Requests by edge- and border nodes
        \end{itemize}
    \item Fabric Edge node supplies first-hop services for users and devices that are connected to the fabric
        \begin{itemize}
            \item Identify and Authenticate
            \item Register endpoint ID with Control Plane
            \item Anycast Layer 3 Gateway (provide the same gateway IP on all Edge Nodes)
            \item De- and encapsulation of Data Traffic 
        \end{itemize}
    \item Fabric Border
        \begin{itemize}
            \item Internal Border (XTR) \\
            Known Routes inside your company \\
            Communicates endpoints to outside and known subnets to inside of the fabric \\
            Hand-off means mapping context (VRF and SGT) between the domains
            \item External Border, Default (PXTR) \\
            Unknown Routes outside your company \\
            \emph{Gateway of last resort} for all unknown destinations \\
            Exports all internal IP ranges as one summarized route to the available IGP \\
            Does not import external Routes
            Hand-off means mapping context (VRF and SGT) between the domains
        \end{itemize}
\end{itemize}

\subsection{Cisco ACI -- Application Centric Infrastructure}
\emph{``... ist eine branchendführende SDN-Lösung, die richtliniengesteuerte Automatisierung über integriertes Underlay und Overlay bietet 
und Richtlinienautomatisierung für VMs, Bare-Metal Server und Container anbietet.''}

\vspace{3mm}
\noindent
Trend: most of the traffic does not leave the datacenter (-> east west traffic), while traditional 3-tier networking model is optimized for north-south traffic.

\noindent
Solution: \textbf{Leaf-Spine Topology}
This architecture is optimized for east-west traffic inside a datacenter, based on Clos Network from 1953.

Every Endpoint is the same distance away from any other endpoint \emph{(deterministic)}

Leaf connect to all and only to spines, spines connect to all and only to leafs.

Scaling is easy: more bandwidth -$>$ add spine / more ports -$>$ add leaf

\vspace{3mm}
\noindent
Trend: Mobility (VMotion), Layer 2 adjacency and ``getting rid of spanning tree''

\noindent
Solution: \textbf{VXLAN} - Creating a tunnel between all devices

\vspace{3mm}
\noindent
Trend: Elasticity, rate of change in a datacenter network is high

\noindent
Solution: Everything is / will be programmable via API!
