\section{Ansible}
Focus in ansible is writing configuration as code that can easily be pushed out to managed devices, 
reducing or eliminating the need to manually configure single devices. 
There is no client installation needed on the endpoints. 

Alternatives to ansible include Puppet, Chef and SaltStack. Based on python and extensible via modules, ansible has about 65\% market share.   

\subsection{Basics and Setup}
Ansible is usually installed on a control node, e.g. an operators workstation. Inventory, code and playbooks can then be managed and shared through version control systems. 
Connections to assets that should be configured are initiated for every task that is run, using the ssh protocol for Linux and winrm for Windows devices..


To run tasks, the following is required on an operators machine:
\begin{itemize}
    \item Python installation
    \item Ansible installation
    \item \ttfamily/etc/hosts \rmfamily for dns resolution
    \item ansible.cfg for configuration (default in \ttfamily/etc/ansible/ansible.cfg\rmfamily) \\ 
    used for default settings including privilege and remote user, ``fallback'' for settings that aren't provided at a more specific level
    \item inventory file (default in \ttfamily/etc/ansible/hosts\rmfamily) \\
     to identify all managed hosts, can be dynamic using central inventory software or static file with host lists
\end{itemize}

\noindent
On the managed devices:
\begin{itemize}
    \item Python installation
    \item Dedicated ansible user account with ssh / sudo access, typically key-based
\end{itemize}

\subsection{Ad-Hoc Commands and Modules}
Executing a single module on some hosts in the inventory can be done by command line: 

\ttfamily
\noindent
ansible 

-i $<$inventory file$>$ all 

-m $<$module name$>$ 

-a $<$module argument$>$ 

-u $<$username$>$ 

-k \rmfamily \emph{(ask for password)}
Running the same task a second time, ansible recognizes if changes had to be done or the command did not have any effect.

Modules to avoid: \emph{command, shell, raw} because there is no idempotency. Ansible cannot determine, if the configuration is already there or has to be applied. If a simple command is run (like useradd), 
instead of recognizing the ``ok state'' (user already exists, do nothing) ansible returns the actual error the command outputs. 
They should only be used as a last resort, if there is no appropriate ansible module available.

\subsection{Playbooks}
A playbook contains multiple tasks that are run in order. Errors in one task will cause the whole play to fail.

Each playbook is  written in yaml. It consists of several p

\subsubsection{Facts and Variables}

\subsubsection{Working with Conditionals}