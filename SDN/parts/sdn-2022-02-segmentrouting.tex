\section{Segment Routing}

RFC 8402: \url{https://datatracker.ietf.org/doc/html/rfc8402}

\emph{Segment Routing (SR) leverages the source routing paradigm.  A node 
   steers a packet through an ordered list of instructions, called 
   "segments".  A segment can represent any instruction, topological or 
   service based.  A segment can have a semantic local to an SR node or 
   global within an SR domain.  SR provides a mechanism that allows a 
   flow to be restricted to a specific topological path, while 
   maintaining per-flow state only at the ingress node(s) to the SR 
   domain.}

\subsection{Source Routing}
The entire path is calculated as a \emph{Segment List} by the source router, or received by a PCE (Path Computation Element).
This path defines either a route inside a topology or some network services (Networks Function Virtualization NFV).

\noindent
Segment = an Instruction
\begin{itemize}
    \item Shortest Path
    \item specific interface
    \item ...
\end{itemize}

\noindent
Each node has one or several \emph{Segment IDs}. 
\begin{itemize}
    \item MPLS Label or IPv6 Address
    \item Global Segment - Each node in the SR domain installs this instruction
    \item Local Segment - only the node that originates this segment installs the associated instruction. \\
    all nodes must know about other nodes' local segments in order to use them.
\end{itemize}

\noindent
A packet header contains an \textbf{SID} as shortest path destination or \textbf{SID List} as defined path.

\noindent
Every hop runs the instructions inside the packet header - there is no per-flow state information. An instruction can be one of the following three.`'
\begin{itemize}
    \item PUSH -- insert segment(s) at the packet head and set first as active
    \item CONTINUE -- active segment is not completed and remains active
    \item NEXT -- active segment is completed, make next item in SID list active
\end{itemize}

\subsection{IGP Extensions to support Segment Routing}
The following segment types must be known to the IGP protocol in order to distribute SIDs. 
\subsubsection{Prefix Segment}
Shortest path to any known network (prefix). Multi hop, equal cost.

\noindent
Prefix SID is domain-wide unique, assigned manually to the loopback address of each node

\noindent
Algorithm id specifies the method of choosing a path - default is 0, shortest path.

\noindent
A Prefix Segment can be of two different types
\begin{itemize}
    \item Node Segment \\
    Associated with a /32 prefix which is a node address. \\
    Sets \textbf{N-Flag} in Segment ID.
    \item Anycast Segment \\
    Associated with an anycast prefix, which routes to the geographically closest out of a group of hosts. \\
    N-Flag is unset! \\ 
    Macro-Engineering: can be used to steer traffic via specific region, or make it pass some router performing special network functions. \\
    Offers ECMP load balancing and high availability.
     
\end{itemize}
\subsubsection{Adjacency Segment \emph{(local)}}
Unidirectional Adjacency, traffic is steered explicitly over an interface / link. Overrides shortest path routing decisions. \\
SID list contains node prefix first, then Adjacency-ID. 
\begin{itemize}
    \item Layer-2 Adjacency can address one specific link inside a Link Aggregation Group (LAG). 
    \item Group Adjacency
\end{itemize} 
\subsubsection{BGP Segments}
\begin{itemize}
    \item BGP Prefix Segment \\
    Global segment, associated with a BGP Prefix \\
    \emph{”steer traffic along the ECMP-aware BGP multi-path to the prefix associated with this segment”} 
    \item BGP Anycast Segment \\
    Traffic steering capabilities such as \emph{”steer traffic via spine nodes in group A”} 
    \item BGP Peer Segment \\
    Associated with BGP Peering sessions to specific neighbor \\ 
    Local segments that are signaled via BGP link-state address-family \\
    \emph{”steer traffic to the specific BGP peer node via ECMP multi-path towards that peer router”} \\
    Overrides the traditional BGP mechanism 
    \item BGP Peer Adjacency Segment \\
    \emph{”steer traffic to the specific BGP peer node via the specified interface towards that peer router”}
\end{itemize}

\noindent
Combining segments can create any kind of end-to-end path.

\noindent
\textbf{Traffic steering} only happens on source nodes to enable per-flow load balancing.

\noindent
For \textbf{traffic engineering} a policy defines the path (SID-List) to be used. 